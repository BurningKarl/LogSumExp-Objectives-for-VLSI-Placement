\begin{otherlanguage}{ngerman}

\chapter{Zusammenfassung} \label{chap:zusammenfassung}

In dieser Bachelor-Arbeit wird ein Teilproblem des Global Placements betrachtet welches beim Chip-Design-Prozess vorkommt.
W\"ahrend ein Chip designt wird, gibt es einen Zeitpunkt zum dem bereits feststeht welche vorgefertigten Bauteile
(bestehend aus einer kleinen Anzahl von Transistoren) f\"ur den Chip ben\"otigt werden und wie diese verbunden werden m\"ussen.
Diese rechteckigen Bauteile werden Zellen genannt und sollen so \"uberlappungsfrei auf der rechteckigen Chipfl\"ache platziert und mit Leiterbahnen verbunden werden,
dass die L\"ange der Leiterbahnen minimal ist, wobei einige Zellen mit vorplatzierten Verbindungsstellen verbunden werden m\"ussen.
Die Verbindungsstellen, die entweder auf den Zellen liegen oder bereits vorplatziert sind, hei\ss{}en Pins
und eine Menge von Pins, die auf dem finalen Chip elektronisch \"aquivalent seien sollen, hei\ss{}t Netz.
Im ersten Schritt, dem Placement, legt man zun\"achst nur die Platzierungen der Zellen fest und minimiert dabei eine Sch\"atzung
der Gesamtverbindungsl\"ange, die eine gewichtete Summe von den gesch\"atzten Verbindungsl\"angen der einzelnen Netze ist.
Erst danach werden die Positionen der Leiterbahnen, die die Pins verbinden, berechnet.

In einer Herangehensweise an das Placement-Problem wird im allerersten Schritt die Bedingung fallen gelassen,
dass sich die Zellen nicht \"uberlappen d\"urfen, und es entsteht ein unbeschr\"anktes Optimierungsproblem.
Im Kontext dieses Optimierungsproblems werden in dieser Bachelor-Arbeit zwei Sch\"atzungen der Netzl\"angen
betrachtet, die auf den Maximumsapproximationen LogSumExp und Weighted-Average basieren.
Diese wurden bereits in anderen Ans\"atzen im Chip-Design erfolgreich eingesetzt und
sind differenzierbare Approximationen der Standard-Netzl\"angen-Sch\"atzung HPWL.
Ihre Differenzierbarkeit macht es m\"oglich konvexe Optimierungsmethoden wie das Gradientenverfahren und Nesterovs Methode anzuwenden,
um sie zu minimieren.

Im ersten Teil (Kapitel \ref{chap:global_placement} bis \ref{chap:theoretical_properties})
wird zun\"achst das Problem in den Kontext des Chip-Designs eingebettet und mathematisch formuliert.
Danach werden die verwendeten Verbindungsl\"angen-Sch\"atzungen definiert
und die beiden konvexen Optimierungsmethoden vorgestellt.
Da nur bestimmte Zielfunktionen kompatibel mit diesen Methoden sind,
werden die analytischen Eigenschaften der zwei Netzl\"angen-Sch\"atzungen (aus denen dann Zielfunktionen werden)
zusammengefasst und etwas erweitert.
Die auf LogSumExp basierende wird mit NLSE abgek\"urzt und die auf Weighted-Average basierende mit NWA.
Beide approximieren HPWL abh\"angig von einem Parameter \(\gamma\) beliebig gut,
wobei NWA bei gleichem \(\gamma\) den geringeren absoluten Fehler hat.
NLSE ist konvex und NWA ist es nicht, obwohl dies in \cite{HsuChangBalabanov-AnalyticalPlacementFor3dIcDesigns} behauptet wird.
NWA verletzt die Konvexit\"atsbedingungen allerdings nur leicht und ist von oben und unten durch eine konvexe Funktion beschr\"ankt.
Zus\"atzlich besitzen beide Netzl\"angen-Sch\"atzungen einen lipschitzstetigen Gradienten.
Diese Eigenschaften gelten ebenfalls f\"ur die Netzl\"angen-Sch\"atzungen wenn man sie als Funktionen der Zellpositionen
anstatt der Pinpositionen betrachtet.

Der zweite Teil (Kapitel \ref{chap:implementation_details} und \ref{chap:results})
k\"ummert sich um die praktische Anwendung dieser Netzl\"angen-Sch\"atzungen im unbeschr\"ankten Optimierungsproblem,
das oben beschrieben wurde.
Nach ein paar Details zur Implementation werden hier NLSE und NWA
mit Hilfe der konvexen Optimierungsmethoden auf mehreren Instanzen
und mit mehreren Parameter-Kombinationen minimiert.
Die Ergebnisse werden danach verglichen mit denen,
die durch das Minimieren von HPWL bzw. quadratischer Netzl\"ange entstehen.
Es stellt sich heraus, dass ein relativ hoher \(\gamma\)-Wert notwendig ist,
damit die Konvergenzeigenschaften von NLSE bzw. NWA gut genug werden.
Leider flacht mit einem hohen \(\gamma\) Wert die Zielfunktion
rund um das Optimum extrem ab und sehr verschiedene Zellplatzierungen
k\"onnen \"ahnlich niedrige Werte der Zielfunktion annehmen.
Au\ss{}erdem kann es selbst mit hohen \(\gamma\)-Werten passieren,
dass die Norm des Gradienten nicht zuverl\"assig w\"ahrend des Optimierungsprozess
zu Null konvergiert, was das Standard-Abbruch-Kriterium unbrauchbar macht.
Fixiert man stattdessen eine gewisse Anzahl an Iterationen, die die Laufzeit in einen akzeptablen Bereich bringen,
so sind die Resultate sowohl in ihrer Qualit\"at als auch in der ben\"otigten Laufzeit,
um diese zu erreichen meist schlechter als Platzierungen, die durch das Minimieren von quadratischer Netzl\"ange entstehen.

\end{otherlanguage}
