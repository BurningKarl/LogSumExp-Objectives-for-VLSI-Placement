\chapter{Introduction} \label{chap:Introduction}

This bachelor thesis is concerned with a subproblem of the global placement problem originating in computer chip design.
Chip design is a field of research that tries to develop methods to automate the design of very-large scale integrated (VLSI) chipsets
because purely manual design would not be able to handle the ever increasing complexity of modern day computer chips.
Many of the challenging problems that arise in this field are only solvable efficiently enough with heuristic algorithms.
Chip design remains an active research area as new heuristics are explored
and some problems change with further progress in chip manufacturing technologies.

At some point during the chip design process it is fixed which prebuilt components will make up the chip and how they are connected.
At this stage the rectangular components called cells need to be placed on the rectangular chip area without overlaps
such that the estimated length of the wires needed to connect the cells is minimal
given that some of the cells need to be connected to previously placed contact points on the chip.
% Mention NP-completeness?
As part of a classical heuristic approach to this problem one relaxes the constraint that the cells may not overlap in the very first step.
This leads to an unconstrained minimization problem that mainly depends on how the wirelength is estimated.
Some wirelength estimations may be minimized very efficiently while others come closer to estimating the real wirelength.
In this bachelor thesis we investigate the use of two wirelength estimations
based on the maximum approximations LogSumExp and Weighted-Average,
which proved to be successful in other contexts, during this unconstrained minimization.
These two wirelength estimations are differentiable approximations of the canonical (and not differentiable) wirelength estimation HPWL
and allow for the use of convex optimization methods such as gradient descent and Nesterov's accelerated gradient method.

This thesis consists of two main parts:
In the first part the problem and the optimization methods are described in detail
and the theoretical properties of the two wirelength estimations are summarized.
Here, the properties that are relevant either to their use as wirelength estimations
or to their use as objective functions that are minimized with convex optimization methods
are presented.
The second part is concerned with some pratical considerations
and the evaluation of the performance of these wirelength estimations
in the unconstrained minmization problem outlined above.
