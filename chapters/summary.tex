\chapter{Summary} \label{chap:summary}

In this bachelor thesis the use of the \(\NLSE\) and \(\NWA\) netlength estimations
in a specific subproblem of the global placement task was investigated.
The theoretical properties of these two netlength estimations were summarized
and expanded to show that the corresponding weighted total wirelength functions
can be used as objective functions for the convex optimization techniques
gradient descent and Nesterov's accelerated gradient method.
Afterwards these convex optimization methods were used in practice on some chip instances
to solve the problem and the results were compared
with approaches that use the classic netlength estimations \(\QCLIQUE\) and \(\HPWL\).

The main takeaways from the theoretical section are the following:
Both \(\NLSE_\gamma\) and \(\NWA_\gamma\) approximate \(\HPWL\)
closer and closer as \(\gamma\) tends to zero
but \(\NWA\) is always closer to \(\HPWL\) for the same \(\gamma\).
\(\NLSE\) is convex and contrary to what was claimed in \cite{HsuChangBalabanov-AnalyticalPlacementFor3dIcDesigns} \(\NWA\) is not convex
but \(\NWA\) only slightly violates the convexity conditions and can be bounded from above and below by a convex function.
Additionally, both netlength estimations have a Lipschitz continuous gradient.
All of these properties also extend to the netlength estimations when viewed as functions
of cell positions rather than pin positions.

In spite of these positive theoretical results, the pratical results are rather disappointing.
The value of \(\gamma\) has to be fairly high in order for the convergence properties
of these netlength estimations to become good enough and then the objective functions
flatten out so much towards their minima that very different placements can achieve close to optimal values.
Even when high \(\gamma\)-values are chosen the convex optimization methods
are not able to reliably decrease the gradient norm towards zero
so that a low gradient norm fails as a good stopping criterion.
If the optimization processes are instead stopped after a fixed number of iterations
that keep the runtime in a reasonable range the resulting placements
are not able to compete with the results of minimizing \(\QCLIQUE\)
in terms of solution quality or runtime.
