\section{Goal} \label{sec:goal}

It is important to understand the goal of investigating the use \(\NLSE\) and \(\NWA\)
during the very first step of global placement.
The goal of the very first step of global placement itself was already stated in \cref{sec:global_placement_and_chip_design}:
Gain insights about the relative positions of the cells in an optimal solution of the Placement task.
Note that all placements with low wirelength (that do not need to obey disjointness constraints)
will have lots of overlapping cells and that the more clumped together the cells are
the fewer clear insights about relative positions can be gained.
Therefore, a balance has to be found between how spread out the cells are and
how low the wirelength is.

When using \(\QCLIQUE\) as the netlength estimation,
the minimization problem can be solved extremely efficiently
and the resulting placement spreads out the cells over large parts of the chip area.
Unfortunately, the real wirelength of a chip depends on the rectilinear distances
between pins and not the quadratic ones.
This can lead to suboptimal solutions.

Minimizing \(\HPWL\) does find placements with very low linear wirelength
but the cells are mostly concentrated on very few spots of the chip area.
If the cells belonging to one of these spots need to be assigned to different regions
during partitioning, the assignment becomes almost arbitrary because the cells are so close to each other.
Additionally, the running time for this approach is considerably higher than
for minimizing \(\QCLIQUE\).

The goal of using the netlength estimations \(\NLSE_\gamma\) and \(\NWA_\gamma\)
is to find a middle ground between \(\QCLIQUE\) and \(\HPWL\) in terms of
approximating linear wirelength while also spreading out the cells on the chip.
At the same time, the minimization processes should of course take a reasonable amount of time.
This goal will guide us in the following when searching for the optimal parameters.
