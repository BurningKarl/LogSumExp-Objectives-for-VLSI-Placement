\chapter{Global Placement} \label{chap:global_placement}

This chapter will give an overview of the global placement problem, its origin and its mathematical formalization.
Afterwards, the netlength estimations that play a role in this thesis are presented.

\section{Global Placement and Chip Design} \label{sec:global_placement_and_chip_design}

Global placement is one of the automated steps in the chip design process.
In order to explain the role of global placement, the following section will be a rough depiction of the chip design process.
For more detailed information about the chip design process we refer to
\cite{AlpertMehtaSapatnekar-HandbookOfAlgorithmsForPhysicalDesignAutomation} and \cite{LavagnoMarkovMartinScheffer-EDAImplementationCircuitDesignProcessTechnology}.

At the start of the chip design process the behavior and performance details of the chip need to be specified.
Then these requirements are translated into a set of prebuilt rectangular components and connections between these components.
The components are called cells, their in- and outputs are called pins 
and the connections are specified in terms of sets of pins that are electrically equivalent and need to be connected.
In one of the last steps prior to production this representation is converted into a geometric model of the final chip
satisfying performance and manufacturing requirements.
This step is called \emph{physical design} and while it is already largely automated
physical design automation remains an active area of research
that aims at further improving the existing algorithms and solving new problems that come up with new advancements in chip technology.

Physical design is further split up into multiple stages.
These stages are not only executed one after the other
but often the results of later stages are used to adapt design decisions which then also influence the earlier stages.
The two major steps that are necessary to understand the core optimization problem of this thesis are \emph{placement} and \emph{routing}.
Placement tries to find valid positions for the components and routing defines the paths of the conductors that connect the pins as specified afterwards.
The placement results heavily influence the results of the routing step and are important for the final performance of the chip.

This leads to the question of how the performance of a chip is measured. 
What is the objective function of placement and routing?
While there are multiple desired traits of a chip such as short cycle times, low power consumption, and low manufacturing cost, 
it is hard to put them in a single objective function and it would be even harder to optimize this objective function.
Therefore, an objective function is used that is easier to handle and correlates with these design goals: 
Total weighted wirelength, i.e. the weighted sum of the lengths of the conductor paths that connect the pins.
Lower wirelength leads to less resistance, less heat and less power consumption and also shorter cycle times.
Weights are introduced to put emphasis on certain connections, for example those that are critical for faster clock cycles.

When choosing wirelength as the objective for placement and routing, a new problem arises.
To compute the total weighted wirelength during placement, we would need to execute a complete routing step and evaluate it afterwards.
This is not feasible since the runtime would be too high.
Instead, the total weighted wirelength is estimated with a simple function that is fast to evaluate.
Taking all those considerations into account, we might formulate the placement problem this way:
Find positions for all cells such that the estimated total weighted wirelength is minimal while the cells do not overlap and meet some additional requirements.

A standard requirement is that the cells need to fit into the power grid that is already laid out on the chip.
The chip area is partitioned into rows that lie between power lines alternating between current supply and ground.
The majority of the cells have a height of exactly one row, these are called standard cells, and some have a height spanning multiple rows, these are called macros.
To correctly fit within the power grid now means that all cell positions have to be aligned with these vertical rows.
Further requirements often include that some cells are only allowed to occupy certain areas of the chip
(some ares might be blocked by previously placed large chip components)
and that the cells need to leave enough free space for the conductor paths.
This shows how complex the placement task can become once all practical demands are formulated.

However, since the placement problem is still too hard to be tackled at once, it is instead split up into three steps:
The first step called \emph{global placement} tries to determine well-spread cell position with little overlap and low estimated wirelength.
The result is then taken as the input to the \emph{legalization} step which tries to legalize it with as little movement of the cells as possible.
Here, a legal placement refers to a placement without any overlaps that also fulfills the additional requirements mentioned above.
Often, an additional optimization step called \emph{detailed placement} is applied to the legal placement to find local improvements not violating any constraints.
In this thesis we will consider a simplification of the global placement task:
Minimization of wirelength estimations without any constraints, in particular without considering cell overlaps.

Solving this problem can be combined with an iterative geometric partitioning approach (see \cite[pp. 12-16]{BrennerVygen-AnalyticalMethodsInVlsiPlacement})
that assigns the cells to smaller and smaller regions of the chip to spread out the cells evenly and reduce overlap.
The idea is to minimize the wirelength estimation for all cells at once first,
then use this placement to assign the cells to regions (for example the four quadrants of the chip) such that the cell density in each region is reasonable,
minimize wirelength estimations in each region, subdivide the regions
and continue alternating between minimization and partitioning until the regions are small enough
for the legalization step.
This means that even though minimizing the netlength estimation for all cells at once will most likely not produce
a placement that is even close to an optimal overlap-free placement,
it should provide useful insights into the relative placements
of the cells, i.e. if a cell should rather be placed on the left half of the chip or on the right half
and if cell A should rather be placed above or below cell B.

\section{Mathematical Description} \label{sec:mathematical_description}

At this point, we will introduce the mathematical formalism necessary to rigorously state the global placement problem.
Note that this description of the global placement problem is a simplification of the what needs to be done in the global placement step as discussed above.
We will mostly follow the notation in \cite{BrennerVygen-AnalyticalMethodsInVlsiPlacement}. 
There are four main ingredients: The set of movable cells \(C\), the set of pins \(P\), the set of nets \(\mathcal{N}\) and the rectangular chip area \([x_{\min}, x_{\max}] \times [y_{\min}, y_{\max}]\).

The cells are rectangular components to be placed on the chip area, but since there are no disjointness constraints we do not care about their shape.
We distinguish between movable and fixed cells: 
Fixed cells are already placed beforehand and cannot be moved
while movable cells are those that we are given the task to determine the position of.
The position of (an anchor point of) a movable cell \(c \in C\) is denoted by \((x(c), y(c)) \in \real^2\).
We also use \(\mathbf{x}^C, \mathbf{y}^C \in \real^C\) as vectors containing all cell positions of movable cells.

Every cell has one or more pins whose relative positions are specified by some offset to the anchor point of the cell.
For fixed cells we may specify the offsets relative to a global anchor point instead of multiple cell-specific anchor points.
Therefore, we assume that there is exactly one fixed cell with coordinates \((0, 0)\) which we will call \(\square\).
To denote that the pin \(p \in P\) belongs to the cell \(c \in C \cup \{\square\}\),
we will write \(\alpha(p) = c\) and the offset of the pin \(p\) relative to the anchor point of \(\alpha(p)\)
is denoted by \((x_{\offs}(p), y_{\offs}(p)) \in \real^2\).

Lastly, \(\mathcal{N}\) is a partition of \(P\) into sets containing at least two elements. 
Each net \(N \in \mathcal{N}\) represents a set of pins that need to be connected on the final chip.
Because not all nets are equally important, a weight \(w(N) \in \real_{\geq 0}\) is assigned to every net \(N\).
The information about cells, pins, pin offsets and nets combined is called the \emph{netlist}.

Let \(L \colon \Set{V \subseteq \real^2 | 2 \leq \abs{V} < \infty} \to \real_{\geq 0}\) be a function that estimates the length of the wires required to connect all pins in a net given their positions (the \emph{netlength}).
Now, the the global placement problem is to find values \(x(c) \in [x_{\min}, x_{\max}]\) and \(y(c) \in [y_{\min}, y_{\max}]\) for each movable cell \(c \in C\) such that the weighted sum of netlengths
\[ \sum_{N \in \mathcal{N}} w(N) L\BktR{\Set{\BktR{x(\alpha(p)) + x_{\offs}(p), \ y(\alpha(p)) + y_{\offs}(p)} \mid p \in N}} \]
is minimized. 

All netlength estimations \(L\) have the property that a translation of all pins in a net leaves the value unchanged.
If there is a connected component in the hypergraph \((P, \mathcal{N})\) without a fixed pin,
there is no unique optimal placement and the choice of an optimal placement of (a part of) the cells will be arbitrary
because all cell positions in this connected component can be shifted simultaneously without changing the value of the objective function.
To avoid this problem, we will only consider those instances of the problem in which each connected component in the hypergraph \((P, \mathcal{N})\)
contains at least one fixed pin. This is typically the case for all practical instances.

Additionally, we assume that \(x_{\offs}(p) \in [x_{\min}, x_{\max}]\) and \(y_{\offs}(p) \in [y_{\min}, y_{\max}]\) for every fixed pin \(p\).
In practice, all pin offsets of pins of movable cells are small and
any selection of cell positions that minimizes the objective function will meet the condition 
that \(x(c) \in [x_{\min} - \varepsilon, x_{\max} + \varepsilon]\) and \(y(c) \in [y_{\min} - \varepsilon, y_{\max} + \varepsilon]\) 
for all \(c \in C\) and some small \(\varepsilon > 0\).
(Moving a cell away from the chip area only increases the distance to every fixed pin.)
Thus, the objective function changes very little if one just moves every cell
that lies outside of the chip area in an optimal solution
to the closest position inside the chip area afterwards and the solution stays close to optimal.
In this sense it suffices to solve the unconstrained minimization problem
and this property is necessary for the convex optimization methods used in this thesis.

For brevity we introduce some additional notation:
Let \(\mathbf{x}^P_{\offs}(N), \mathbf{y}^P_{\offs}(N) \in \real^N\) be the vectors containing the pin offsets of the pins in a net \(N\)
and \(A(N) \in \real^{N \times C}\) be the matrix that maps cells to pins of \(N\) by setting \((A(N))_{p, c} = 1\) if \(\alpha(p) = c\) and 0 otherwise.
Multiplying a vector with entry 1 at position \(c \in C\) with this matrix gives a vector with entry 1 at each pin of \(N\) belonging to cell \(c\).
These definitions allow us to express the pin positions \(\mathbf{x}^P(N), \mathbf{y}^P(N) \in \real^N\) in \(N\) 
as an affine linear transformation of the cell positions:
\[ \mathbf{x}^P(N) = A(N) \mathbf{x}^C + \mathbf{x}_{\offs}^P(N) \quad \text{and} \quad \mathbf{y}^P(N) = A(N) \mathbf{y}^C + \mathbf{y}_{\offs}^P(N) \]
By defining \(L(\mathbf{x}, \mathbf{y})\) as a shorthand for \(L(\set{ (x_i, y_i) | 1 \leq i \leq n })\) for \(\mathbf{x}, \mathbf{y} \in \real^n\),
we can write the objective function of the global placement problem as
\[T_L(\mathbf{x}, \mathbf{y}) \deq \sum_{N \in \mathcal{N}} w(N) L \BktR{A(N) \mathbf{x} + \mathbf{x}_{\offs}^P(N), A(N) \mathbf{y} + \mathbf{y}_{\offs}^P(N)}\]

At this point, it is clear that there is a difference between a netlength estimation \(L\) and its corresponding single net objective function.
While \(L\) estimates the netlength based on the pin positions,
the corresponding objective function estimates the netlength based on the cell positions (using \(L\)).
Whenever we mean the objective function for a net \(N\), we will use
\[ L^N(\mathbf{x}, \mathbf{y}) = L \BktR{A(N) \mathbf{x} + \mathbf{x}_{\offs}^P(N), A(N) \mathbf{y} + \mathbf{y}_{\offs}^P(N)} \]
as an abbreviation.



\section{Netlength Estimation} \label{sec:netlength_estimation}

An important factor for the performance of any algorithm that aims to solve the global placement problem is the choice of the function that estimates the netlength.
In this thesis, our goal is to compare different netlength estimations and their optimization methods.
This section should serve as an overview of the estimations in this thesis as well as some important ones in practice and theory.

When routing a chip, i.e. laying out all the conductor paths on the chip, these paths can only go horizontally or vertically, never diagonally.
In other words, the rectilinear distance is used to measure the length of a conductor path between two pins on the chip.
Because of this property, some functions that are estimating the netlength are a sum of the estimated netlength in x direction
and the estimated netlength in y direction.
Of course, there is no substantial difference between those directions so these functions (most often) are of the form:
\( L(\mathbf{x}, \mathbf{y}) = \overline{L}(\mathbf{x}) + \overline{L}(\mathbf{y}) \).
This property allows us to optimize x- and y-values independently from one another allowing for parellel execution of the optimization processes.

We will also use property of the netlength approximations to shorten the notation in this thesis. 
Whenever there is a bar above a function \(L\) estimating the netlength it refers to the one-dimensional part of \(L\)
and the corresponding function for two dimensions is \(\overline{L}(\mathbf{x}) + \overline{L}(\mathbf{y})\).
To be well-defined, we set \(\overline{L}^N\) to be the one-dimensional single net objective function in x-direction
but this notation will only be used in a general setting in which this decision does not matter.

\subsection{Rectilinear Steiner Trees} \label{subsec:rectilinear_steiner_trees}
As already mentioned, conductor paths can only go horizontally or vertically.
So when routing a single net, the lowest total wirelength can be achieved by routing with a shortest rectilinear Steiner tree
where the terminals are the pin positions.
This makes the length of a shortest rectilinear Steiner tree an obvious starting point for netlength estimation.
We will denote this length as \(\STEINER(\mathbf{x}, \mathbf{y})\).
Unfortunately, finding the length of a shortest rectilinear Steiner tree for a given set of points in the plane is NP-hard as shown in \cite{GareyJohnson-RectilinearSteinerTree}.
Because algorithms to determine a shortest (or even just sufficiently short) rectilinear Steiner tree are often very complex and time-consuming,
this netlength estimation is not often used in practice.
Also, this function will not be discussed in this thesis except for some theoretical comparisons.

\subsection{Half-Perimeter Wirelength} \label{subsec:hpwl}
The \emph{half-perimeter wirelength} (HPWL) or \emph{bounding box} heuristic estimates the netlength as half the perimeter of the bounding box of the pin positions:
\begin{align*}
 \HPWL(\mathbf{x}, \mathbf{y}) &\deq \max_{i, j \in \{1, \ldots, n\}} \abs{x_i - x_j} + \max_{i, j \in \{1, \ldots, n\}} \abs{y_i - y_j} \\
                               &= \max(\mathbf{x}) - \min(\mathbf{x}) + \max(\mathbf{y}) - \min(\mathbf{y})
\end{align*}
The advantage of this expression is that it is easy to write and fast to evaluate.
For nets with two or three pins the HPWL coincides with the length of a rectilinear Steiner tree.
The HPWL is the standard metric to compare the wirelength of placements in benchmarks suites like the 
ISPD 2005 \cite{NamAlpertVillarrubiaWinterYildiz-ISPD2005BenchmarkSuite} and ISPD 2006 \cite{Nam-ISPD2006BenchmarkSuite}.
This function is not differentiable thus excluding analytical methods as optimization methods 
but the minimization of weighted wirelength with HPWL as netlength estimation can be formulated as an LP or a MinCostFlow problem
as shown in \cite[pp. 5-7]{BrennerVygen-AnalyticalMethodsInVlsiPlacement}.

\subsection{Quadratic Wirelength} \label{subsec:qp}
The idea of quadratic wirelength is to replace the rectilinear distance function
\[d_L((x_p, y_p), (x_q, y_q)) = \abs{x_p - x_q} + \abs{y_p - y_q}\]
which is not differentiable by a quadratic distance function 
\[d_Q((x_p, y_p), (x_q, y_q)) = \BktR{x_p - x_q}^2 + \BktR{y_p - y_q}^2 .\]
Because a distance function can not handle more than two points a netmodel is used to break down a net with more than two pins into two pin nets.
A classic example of such a netmodel is the CLIQUE net model.
It transforms a net \(N\) into \(\binom{\abs{N}}{2}\) two pin nets by creating a new net for each pair of pins.
To keep large nets from dominating the objective function, each new net is weighted by \(\frac{1}{\abs{N}-1}\).
The resulting netlength estimation is
\[ \QCLIQUE(\mathbf{x}, \mathbf{y}) \deq \frac{1}{\abs{N}-1} \sum_{i, j \in \{1, \ldots, n\}} (x_i - x_j)^2 + (y_i - y_j)^2 . \]
For an overview of different net models and quadratic wirelength we refer to \cite{BrennerVygen-AnalyticalMethodsInVlsiPlacement}.
Minimizing quadratic wirelength is, for example, used in GORDIAN \cite{KleinhansSiglJohannesAntreich-GORDIAN} and BonnPlace \cite{BrennerStruzyna-GlobalPlacement}.
Replacing the distance function comes at the cost of a skewed objective function which might lead to suboptimal results when evaluated with HPWL.

\subsection{LogSumExp} \label{subsec:nlse}
The usage of the \emph{LogSumExp} (LSE) function in placement was proposed in \cite{NaylorDonellySha-USPatent} and was used in multiple nonlinear placers 
such as mPL6 \cite{ChanCongShinnerlSzeXie-MPL6}, NTUplace3 \cite{ChenJiangHsuChenChang-NTUplace3} and APlace \cite{KahngWang-APlace}.
The LSE function itself is defined as 
\[ \LSE_{\gamma}(\mathbf{x}) = \gamma \log \BktR{\sum_{i = 1}^n \exp \BktR{\frac{x_i}{\gamma}}}, \quad \gamma > 0 \]
for all \(\mathbf{x} \in \real^n\) and approximates the maximum function.
Because the HPWL can be written solely in terms of the maximum function,
we may use the \(\LSE\) function to get a differentiable approximation of HPWL
which we will call \emph{Netlength-LogSumExp} (NLSE):
\[ \overline{\NLSE}_{\gamma}(\mathbf{x}) \deq \LSE_{\gamma}(\mathbf{x}) + \LSE_{\gamma}(-\mathbf{x}) \approx \max(\mathbf{x}) + \max(-\mathbf{x}) = \overline{\HPWL}(\mathbf{x}) .\]

\subsection{Weighted-Average} \label{subsec:nwa}
A close relative to the LSE function is the \emph{Weighted-Average} (WA) function defined as
\[ \WA_{\gamma}(\mathbf{x}) = \frac{\sum_{i = 1}^n x_i \exp(x_i / \gamma)}{\sum_{i = 1}^n \exp(x_i / \gamma)}, \quad \gamma > 0 \]
which has been proposed in \cite{HsuChangBalabanov-AnalyticalPlacementFor3dIcDesigns} and used in ePlace \cite{LuChenChangShaHuangTengCheng-ePlace}.
It approximates the maximum function even better than LSE in terms of absolute error and can also be used to get a differentiable approximation of HPWL 
which we will call \emph{Netlength-Weighted-Average} (NWA) in the same way as above:
\[ \overline{\NWA}_{\gamma}(\mathbf{x}) \deq \WA_{\gamma}(\mathbf{x}) + \WA_{\gamma}(-\mathbf{x}) \approx \max(\mathbf{x}) + \max(-\mathbf{x}) = \overline{\HPWL}(\mathbf{x}) .\]

\subsection{A Visual Comparison} \label{subsec:visual_netlength_comparison}

The previous description mostly consists of formulas.
To get a direct visual comparison, \cref{fig:one_dimensional_netlength_comparison}
graphs all of the netlength estimations above (except for STEINER)
for a net with one fixed pin at coordinates \((0, 0)\) and one movable pin at coordinate \((x, 0)\).
The STEINER netlength estimation is omitted because it coincides with the HPWL netlength estimation for a two-pin net.

\begin{figure}[ht]
 \centering
 \begin{tikzpicture}
  \newcommand\XRANGE{4.5}
  \begin{axis}[
    width=\textwidth-1cm,
    height=(\textwidth-1cm)/2,
%     axis lines=left,
    xmin=-\XRANGE, xmax=\XRANGE,
    ymin=0, ymax=\XRANGE,
    legend pos=south west,
    every axis plot/.append style={
      domain=-(\XRANGE + 1):(\XRANGE + 1),
      samples=201,
      ultra thick,
    },
    xlabel=\(x\),
    ylabel=Netlength estimation,
  ]
    \addlegendentry{\(\HPWL\)}
    \addplot[red]
      {abs(x)};

    \addlegendentry{\(\QCLIQUE\)}
    \addplot[blue]
      {x^2};

    \addlegendentry{\(\NLSE_1\)}
    \addplot[SpringGreen]
      {ln(exp(0) + exp(x)) +
	ln(exp(0) + exp(-x))};

    \addlegendentry{\(\NWA_1\)}
    \addplot[OliveGreen]
      {(0*exp(0) + x*exp(x))/(exp(0) + exp(x)) +
	(0*exp(0) - x*exp(-x))/(exp(0) + exp(-x))};           
  \end{axis}
 \end{tikzpicture}
 \caption{One-dimensional netlength estimations for a net with one fixed and one movable pin}
 \label{fig:one_dimensional_netlength_comparison}
\end{figure}
